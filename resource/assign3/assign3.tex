\documentclass[11pt, oneside]{article}   	% use "amsart" instead of "article" for AMSLaTeX format
%\usepackage{geometry}                		% See geometry.pdf to learn the layout options. There are lots.
%\geometry{letterpaper}                   		% ... or a4paper or a5paper or ... 
%\geometry{landscape}                		% Activate for rotated page geometry
%\usepackage[parfill]{parskip}    		% Activate to begin paragraphs with an empty line rather than an indent

\usepackage{geometry}
 \geometry{
 a4paper,
 total={170mm,257mm},
 left=20mm,
 top=25mm,
 bottom=25mm
 }

\usepackage{graphicx}				% Use pdf, png, jpg, or eps§ with pdflatex; use eps in DVI mode
								% TeX will automatically convert eps --> pdf in pdflatex		
\usepackage{amssymb}
\usepackage{amsmath}
\usepackage{fancyhdr}
\usepackage[utf8]{inputenc}
\usepackage[english]{babel}
\usepackage{enumerate}
\usepackage{arcs}
\usepackage{cancel}
\usepackage{xfrac}
\usepackage{amsthm}
\usepackage{gensymb}
\usepackage{xspace}
\usepackage{hyperref}
\usepackage{fancyvrb}
\usepackage{subcaption}
%\usepackage{fontawesome}


\usepackage{listings}
\usepackage[most]{tcolorbox}
%\usepackage{inconsolata}

\newtcblisting[auto counter]{sexylisting}[2][]{sharp corners, 
    fonttitle=\bfseries, colframe=gray, listing only, 
    listing options={basicstyle=\ttfamily,language=java}, 
    title=Listing \thetcbcounter: #2, #1}

%\usepackage{ctex}

%SetFonts

%SetFonts

\usepackage[inline]{asymptote}


\usepackage[framemethod=tikz]{mdframed}

\newtheorem{example}{Example}
\mdfdefinestyle{examp}{
  linecolor=cyan,
  backgroundcolor=yellow!20
  % , rotatebox
}
\surroundwithmdframed[style=examp]{example}

\usepackage{environ}
\NewEnviron{Example}
{%
\noindent
\begin{minipage}[t]{\linewidth}
\begin{example}
\BODY
\end{example}%
\end{minipage}%
}%

\newtheorem*{solution}{Solution}
\mdfdefinestyle{sol}{
  linecolor=red,
  % , rotatebox
}
\surroundwithmdframed[style=sol]{solution}

\usepackage{environ}
\NewEnviron{Solution}
{%
\noindent
\begin{minipage}[t]{\linewidth}
\begin{solution}
\BODY
\end{solution}%
\end{minipage}%
}%


\pagestyle{fancy}
\fancyhf{}
\lhead{\leftmark}
\cfoot{\thepage}

\title{Assignment 3: The Robust Visual Inertial Odometry (ROVIO) Framework}
\author{Zifei (David) Zhong}
\date{CSCE 774, Fall 2023}					

\newcommand{\latex}{\LaTeX\xspace}


\begin{document}
\maketitle
The purpose of this assignment is to install and make the ROVIO
(Robust Visual Inertial Odometry) framwork work with datasets provided
by this class.

\section{System Environment}
Since the system environment signficantly affects the building and
running of the ROVIO package, I here list the system environment in
Table~\ref{tab.sys}. 

\begin{table}[ht]
  \centering
  \begin{tabular}{|c|c|}
    \hline
    System/Software & Version \\
    \hline
    \hline
    Operating System & Ubuntu 22.04 \\
    \hline
    ROS & Noetic \\
    \hline
    Python & 3.8 \\
    \hline 
  \end{tabular}
  \caption{System Environment}
  \label{tab.sys}
\end{table}

\section{Environment Setup}
It could take several days to get the system environment setup
successfully. Here I provide the setup steps so that this report can
be used as a reference in the future to facilite setting up system
environment.

\subsection{Step 1: Python Virtual Environment}
Setup Python virtual environment so that the environment is
independent for this project.

\begin{verbatim}
sudo apt install python3.8-venv
python3 -m venv tutorial_env
source tutorial_env/bin/activate
\end{verbatim}


\subsection{Step2 2: Install ROS Noetic}
Install ROS Noetic ollowing the installation guide from this web:\\
\verb+http://wiki.ros.org/noetic/Installation/Ubuntu+

After successfully installing ROS Noetic, we need to make sure that
\verb+catkin build+ works well on our system. Following the
instructions on this web to build a workspace:\\
\verb+https://catkin-tools.readthedocs.io/en/latest/verbs/catkin_build.html+


\begin{verbatim}
export ROS_DISTRO=noetic                                 # Set ROS distribution
mkdir -p ros_tutorials_ws/src                            # Create workspace
cd ros_tutorials_ws/src                                  # Navigate to source space
rosinstall_generator --deps ros_tutorials > .rosinstall  # Get list of packages
wstool update                                            # Checkout all packages
cd ros_tutorials_ws                                # Navigate to ros workspace root
catkin init                                        # Initialize workspace
catkin build                                       # Build workspace
\end{verbatim}

Once you see the following messages, the build is successful.
\begin{verbatim}
[build] Summary: All 40 packages succeeded!
[build]   Ignored:   None.
[build]   Warnings:  None.
[build]   Abandoned: None.
[build]   Failed:    None.
[build] Runtime: 2 minutes and 53.9 seconds total.
[build] Note: Workspace packages have changed, please re-source setup
        files to use them.
\end{verbatim}

There are totally 40 packages to be built. If you build these 40
packages successuflly, your system setup is completed. This will save
us a great amount to time to build the ROVIO packages.

\subsection{Step 3: Install Rovio/Kindr}
Get ROVIO source code from this git repo:\\
\begin{verbatim}
https://github.com/ethz-asl/rovio
\end{verbatim}
and follow the link from the web to get code for ``kindr''.

\begin{verbatim}
cd tutorial_env/src
git clone git@github.com:ethz-asl/kindr.git
git clone git@github.com:ethz-asl/rovio.git

cd tutorial_env/src/rovio
git submodule update --init --recursive

cd tutorial_env/src
catkin build rovio --cmake-args -DCMAKE_BUILD_TYPE=Release -DMAKE_SCENE=ON
\end{verbatim}

We got several errors while building \verb+rovio+. We provided
solutions to fixing these errors in the Troublshooting section.

\section{Troubleshooting}
\subsection{Build Error 1}
When \verb+catkin build+ the workspace, we encountered the following
problem:
\begin{verbatim}
/usr/bin/ld: src.c:(.text+0x63): undefined reference to `pthread_join'
collect2: error: ld returned 1 exit status
\end{verbatim}
To solve the above problem, do
\begin{verbatim}
pip3 install git+https://github.com/catkin/catkin_tools.git
\end{verbatim}

\subsection{Build Error 2}
Most likely you will also experience the following problems as I did:
\begin{verbatim}
can't find '__main__' module in '.../lib/python3/dist-packages/em'
\end{verbatim}

or
\begin{verbatim}
AttributeError: 'module' object has no attribute 'RAW_OPT'
\end{verbatim}

To solve the above problems, do
\begin{verbatim}
pip3 uninstall em
pip3 install empy==3.34
\end{verbatim}

Since we have a virtual enviroment, the installing/uninstalling
packages only affects our virtual environment. It would not affect
other projects on our computer, neither would it affected by the
packages we installed for other projects.

\subsection{Build Error 3}
We got the following error when doing catkin build for rovio:
\begin{verbatim}
.../rovio/ImgUpdate.hpp:590:68: error: ‘CV_GRAY2RGB’ was not declared in this scope
  590 | cvtColor(meas.aux().pyr_[i].imgs_[0], filterState.img_[i], CV_GRAY2RGB);
\end{verbatim}

The fix is to edit the line 590 of file \verb+ImgUpdate.hpp+, changing
it to:
\begin{verbatim}
#if (CV_VERSION_MAJOR >= 4)
cvtColor(meas.aux().pyr_[i].imgs_[0], filterState.img_[i], cv::COLOR_GRAY2RGB);
#else
cvtColor(meas.aux().pyr_[i].imgs_[0], filterState.img_[i], CV_GRAY2RGB);
#endif
\end{verbatim}
The reason was that we had \verb+libopencv+ 4.2 in our Ubuntu system.


\subsection{Build Error 4}
We also got the following error when building \verb+rovio+:
\begin{verbatim}
...
/usr/bin/ld: .../librovio.so: undefined reference to `__glewUniformMatrix4fv'
/usr/bin/ld: .../librovio.so: undefined reference to `__glewUniform1i'
/usr/bin/ld: .../librovio.so: undefined reference to `__glewAttachShader'
/usr/bin/ld: .../librovio.so: undefined reference to `__glewCreateShader'
/usr/bin/ld: .../librovio.so: undefined reference to `glewInit'
collect2: error: ld returned 1 exit status
\end{verbatim}

The fix is to search for  \verb+${GLEW_LIBRARY}+  in the file
\verb+../src/rovio/CMakeLists.txt+, and then change it to
\verb+${GLEW_LIBRARIES}+.


\section{Running ROVIO with EuRoC datasets}
We used the EuRoC datasets to test ROVIO framework. The datasets are
available at:\\
\verb+https://projects.asl.ethz.ch/datasets/doku.php?id=kmavvisualinertialdatasets+

The 6 datasets we used to test ROVIO are presented in
Table~\ref{tab.datasets}. The screen shots of running the 6 datasets
are presented in Figure~\ref{fig.datasets}.
\begin{table}[ht]
  \centering
  \begin{tabular}{|c|c|}
    \hline
    Dataset Name & File Name \\
    \hline
    \hline
    Machine Hall 01 & MH\_01\_easy.bag \\
    \hline
    Machine Hall 03 & MH\_03\_medium.bag \\
    \hline
    Machine Hall 05 & MH\_05\_difficult.bag \\
    \hline
    Vicon Room 1 01 & V1\_01\_easy.bag\\
    \hline
    Vicon Room 1 02 & V1\_02\_medium.bag\\
    \hline
    Vicon Room 1 03 & V1\_03\_difficult.bag\\
    \hline
  \end{tabular}
  \caption{EuRoC datasets used in testing}
  \label{tab.datasets}
  \end{table}


\begin{figure}[ht]
\centering
\begin{subfigure}[b]{0.3\textwidth}
\includegraphics[width=\textwidth]{imgs/mh_easy_01.png}
\caption{Machine Hall (easy)}
\end{subfigure}
\begin{subfigure}[b]{0.3\textwidth}
\includegraphics[width=\textwidth]{imgs/mh_medium_03.png}
\caption{Machine Hall (medium)}
\end{subfigure}
\begin{subfigure}[b]{0.3\textwidth}
\includegraphics[width=\textwidth]{imgs/mh_difficult_05.png}
\caption{Machine Hall (difficult)}
\end{subfigure}\\
\begin{subfigure}[b]{0.3\textwidth}
\includegraphics[width=\textwidth]{imgs/v1_easy_01.png}
\caption{Vicon Room 1 (easy)}
\end{subfigure}
\begin{subfigure}[b]{0.3\textwidth}
\includegraphics[width=\textwidth]{imgs/v1_medium_02.png}
\caption{Vicon Room 2 (medium)}
\end{subfigure}
\begin{subfigure}[b]{0.3\textwidth}
\includegraphics[width=\textwidth]{imgs/v1_difficult_03.png}
\caption{Vicon Room 3 (difficult)}
\end{subfigure}
\caption{Screen shots from running ROVIO with 6 EuRoC datasets}
\label{fig.datasets}
\end{figure}


For future reference, running ROVIO with dataset bag file:
\verb+MH_01_easy.bag+, we had the following the terminal output:
\begin{verbatim}
process[rovio-2]: started with pid [59979]
Set Camera Matrix to:
458.654       0 367.215
      0 457.296 248.375
      0       0       1
Set distortion parameters (Radtan) to: k1(-0.283408), k2(0.0739591),
                k3(0), p1(0.00019359), p2(1.76187e-05)
Testing Prediction
==== Test successful (7.8358e-08) ====
==== Test successful (6.75411e-08) ====
Testing cameraOutputCF
==== Test successful (6.66134e-08) ====
Testing imuOutputCF
==== Test successful (1.27552e-08) ====
Testing attitudeToYprCF
==== Test successful (6.43838e-08) ====
Testing transformFeatureOutputCT
==== Test successful (1.44278e-08) ====
Testing LandmarkOutputImuCT
==== Test successful (2.5853e-08) ====
Testing FeatureOutputReadableCT
==== Test successful (1.47619e-08) ====
Testing pixelOutputCT (can sometimes exhibit large absolut errors due to the float precision)
==== Model jacInput Test failed: 74.0615 is larger than 1 at row 1(def_0) and col 0() ====
  -72871.9  -72797.9
Testing zero velocity update
==== Test successful (1.39778e-10) ====
==== Test successful (1.39778e-10) ====
Recording
Storing output to: .../ros_tutorial_ws/dataset/rovio/2023-12-16-22-36-20_25_4_6_1_0.bag
-- Filter: Initializing using accel. measurement ...
-- Filter: Initialized at t = 1403636579.76
[rovio-2] process has finished cleanly
log file: /home/david/.ros/log/881f2e6c-9c63-11ee-83d9-85675d37fb69/rovio-2*.log
\end{verbatim}


\section{Running ROVIO with new datasets}
We would like to run ROVIO with the newly collected datasets and
compare the results generated by other packages.

We need to modify the code in some ways such that it can take our newly
collected videos as input and compute the trajectory. 

We will continue this work and hopefully present the result as a
submission to a research conference.
\end{document} 
